\documentclass[a4paper, 12pt]{book}
%\usepackage[active]{srcltx}
\usepackage[czech]{babel}
\usepackage[utf8]{inputenc}

\usepackage{amsmath}
\usepackage{amsfonts}
\usepackage{amssymb}
\usepackage{amsthm}
\usepackage{mathtools} % needs texlive-latex3 package (Ubuntu)

\usepackage{graphicx}
\usepackage{hyperref}
\usepackage[vlined]{algorithm2e}
\hypersetup{backref,colorlinks=true, raiselinks=true}
%
\theoremstyle{definition}
\newtheorem{theorem}{Věta}[section]
\newtheorem{proposition}[theorem]{Tvrzení}
\newtheorem{definition}[theorem]{Definice}
\newtheorem{remark}[theorem]{Poznámka}
\newtheorem{lemma}[theorem]{Lemma}
\newtheorem{corollary}[theorem]{Důsledek}
\newtheorem{example}[theorem]{Příklad}
\newtheorem{exercise}[theorem]{Cvičení}

%\numberwithin{equation}{document}
%
\def\div{{\rm div}}
\def\Lapl{\Delta}
\def\grad{\nabla}
\def\supp{{\rm supp}}
\def\dist{{\rm dist}}
%\def\chset{\mathbbm{1}}
\def\chset{1}
%
\def\Tr{{\rm Tr}}
\def\to{\rightarrow}
\def\weakto{\rightharpoonup}
\def\imbed{\hookrightarrow}
\def\cimbed{\subset\subset}
\def\range{{\mathcal R}}
\def\leprox{\lesssim}
\def\argdot{{\hspace{0.18em}\cdot\hspace{0.18em}}}
\def\Distr{{\mathcal D}}
\def\calK{{\mathcal K}}
\def\FromTo{|\rightarrow}
\def\convol{\star}
\def\impl{\Rightarrow}
\DeclareMathOperator*{\esslim}{esslim}
\DeclareMathOperator*{\esssup}{ess\,supp}
\DeclareMathOperator{\ess}{ess}
\DeclareMathOperator{\osc}{osc}
\DeclareMathOperator{\curl}{curl}
\DeclareMathOperator{\cotg}{cotg}
\def\Lin{\mathit{Lin}}
\def\dx{\d x}
%
%\def\Ess{{\rm ess}}
%\def\Exp{{\rm exp}}
%\def\Implies{\Longrightarrow}
%\def\Equiv{\Longleftrightarrow}
% ****************************************** GENERAL MATH NOTATION
\def\Real{{\rm\bf R}}
\def\Rd{{{\rm\bf R}^{\rm 3}}}
\def\RN{{{\rm\bf R}^N}}
\def\D{{\mathbb D}}
\def\Nnum{{\rm\bf N}}
\def\Qnum{{\rm\bf Q}}
\def\Measures{{\mathcal M}}
\def\d{\,{\rm d}}               % differential
\def\sdodt{\genfrac{}{}{}{1}{\rm d}{{\rm d}t}}
\def\dodt{\genfrac{}{}{}{}{\rm d}{{\rm d}t}}
%
\def\vc#1{\mathbf{\boldsymbol{#1}}}     % vector
\def\tn#1{{\mathbb{#1}}}    % tensor
\def\abs#1{\lvert#1\rvert}
\def\Abs#1{\bigl\lvert#1\bigr\rvert}
\def\bigabs#1{\bigl\lvert#1\bigr\rvert}
\def\Bigabs#1{\Big\lvert#1\Big\rvert}
\def\ABS#1{\left\lvert#1\right\rvert}
\def\norm#1{\bigl\Vert#1\bigr\Vert} %norm
\def\close#1{\overline{#1}}
\def\inter#1{#1^\circ}
\def\eqdef{\mathrel{\mathop:}=}     % defining equivalence
\def\where{\,|\,}                    % "where" separator in set's defs
\def\timeD#1{\dot{\overline{{#1}}}}
%
% ******************************************* USEFULL MACROS
\def\RomanEnum{\renewcommand{\labelenumi}{\rm (\roman{enumi})}}   % enumerate by roman numbers
\def\rf#1{(\ref{#1})}                                             % ref. shortcut
\def\prtl{\partial}                                        % partial deriv.
\def\Names#1{{\scshape #1}}
\def\rem#1{{\parskip=0cm\par!! {\sl\small #1} !!}}
\def\vysl#1{\par$[$ #1 $]$}
\def\df#1{\emph{#1}}
\def\todo#1{{\color{green}TODO:} #1}

%\newenvironment{pmatrix*}[1][\unskip]{\left(\begin{array}[ #1]}{\end{array}\right)}

%
%
% ******************************************* DOCUMENT NOTATIONS
% document specific
%***************************************************************************
%
\addtolength{\textwidth}{2cm}
\addtolength{\vsize}{2cm}
\addtolength{\topmargin}{-1cm}
\addtolength{\hoffset}{-1cm}
\begin{document}
\parskip=2ex
\parindent=0pt

\section{Basic equations}
The resistivity method use multipoint measurements of the ground (or the media) resistivity in order to reconstruct 
the resistivity spatial distribution (tomography) which in turn may indicate geological structure. The resistivity is related to
the geological parameters as: the water content, the mineral content, other fluids, porosity.

Essential equation is the Ohm's law:
\[
    \vc J = \sigma \vc E
\]
where $\vc J$ is the electric current density $[]$, $\sigma$ is the conductivity $[]$ and $E$ is the electric field intensity $[Vm^{-1}]$.
The resistivity $\rho=1/\sigma$. Intensity is potential field, thus:
\[
    \vc E = -\grad u
\]
where $u$ is the electric potential $[V]$
Combinig these with conservation of the electric charge, we get:
\[
    \div \vc J = f
\]
where $f$ is the density of electric sources, usually two or more point sources.


First, let us consider a simplest case whole space of homogeneous material with a single point source
\[
    f(\vc x)=I\delta(\vc x - \vc x_0)
\]
located at point $\vc x_0$. We get the radial solution:
\[
    u(\vc x) = \frac{\rho I}{4\pi |\vc r|}, \quad \vc r = \vc x - \vc x_0.
\]
This can be derived as a limit case of a oter domain of a sphere of radius $R$ and 
taking limit as $R\to 0$. This also justifies the scaling factor $1/4\pi$. 

For the case of a halfspace domain with homogeneous Neuman BC we get:
\[
    u(\vc x) = \frac{\rho I}{2\pi |\vc r|}, \quad \vc r = \vc x - \vc x_0.
\]


\section{Resistivity mesurement}
Practical setup use $4$ electrodes: two current electrodes $C1$, $C2$  and two potential electrodes $P1$, $P2$.
The constant current $I$ is mainteined between $C1$ an $C2$ while the potential difference $du$ is measured between $P1$, $P2$.'s


Assuming a zero current through the potential electrodes, we have solution:
\[
    u(\vc x) = \frac{I\rho}{2\pi} \Big( \frac{1}{|\vc x - \vc x_{C1}|} - \frac{1}{|\vc x - \vc x_{C2}|}\Big).
\]
Then for the potential difference we have:
\[
    du = \frac{I\rho}{2\pi} \Big( \frac{1}{|\vc x_{P1} - \vc x_{C1}|} - \frac{1}{|\vc x_{P1} - \vc x_{C2}|}
    -\frac{1}{|\vc x_{P2} - \vc x_{C1}|} + \frac{1}{|\vc x_{P2} - \vc x_{C2}|} \Big).
\]
or
\[
    du = \frac{I\rho}{2\pi} \Big( \frac{1}{r_{P1C1}} - \frac{1}{r_{P1C2}} - \frac{1}{r_{P2C1}} + \frac{1}{r_{P2C2}} \Big) = \frac{I\rho}{k}.
\]
where $k$ is a geometry factor.

Considering the material homogeneous we can determine \emph{apparent resistivity} as:
\[
    \rho = k\frac{du}{I}.
\]

\subsection{Aparent resistivity, general geometry}
Apparent resistivity is computed as:
\[
    \rho_a = \frac{\overline{du}}{I\, dU}, \quad dU = U(P2) - U(P1), 
\]
where $U$ is solution to the problem:
\[
   -\Lapl U = \delta_{C2} - \delta_{C1}.
\]




\section{Inversion problems}


\subsection{Tichonov regularization}
Techniques used in inversion problems are very general and can be applied to wide range of forward models. We consider a general inverse problem
to find a continuous and piecewise smooth field $m(\vc x)$, $\vc x\in \Omega$ such that the forward model $\vc F[m]$ match the observation vector $\overline{\vc F}$.

The field parameter $m$ is found  as a minimizer of:
\[
    \Phi[m] = \Phi_m[m] + \beta\Phi_r[m].
\]
The cost functional consists of the misfit cost:
\[
    \Phi_m[m] = \sum_{i\in I} \frac{1}{\epsilon_i^2} |F_i[m] - \overline{F}_i|^2 = |\tn D (\vc F[m] - \overline{\vc F})|_2^2
\]
where $\epsilon_i$ is standard deviance of a single obseravation $i$ from set of observations $I$, this forms a weight diagonal matrix $D_{i,i} = 1/\epsilon_i$.

As the problem is vastly underdetermined, we have to use a suitable regularization functional that reflects our appriori knowledge about the field $m$:
\[
    \Phi_r[\vc m] = \norm{ m - m_0}_{2,\Omega}^2 + \int_\Omega \grad (m - m_0)^T \tn K \grad (m - m_0).
\]
This regularization prescribes the appriori field values $m_0$ and penalizes the gradients. The tensor field $\tn K$ can be used for:
\begin{itemize}
 \item take less smoothing at places of possible discontinuities
 \item smooth more far from the measurement
 \item count for anisotropy, e.g. in the case of 1D measurement in a tunnel we use cylindric coordinates and
 penalize changes in $\phi$ coordinate.
\end{itemize}
In principle $\tn K$ can control the regularization better then a single regularization parameter $\beta$ so we should
find suitable methods how to change $\tn K$ e.g. to find discontinuities automatically.

The field $m$ is assumed in a form of linear combination:
\[
    m(\vc x) = \sum_{l=1}^L m_l \psi_l(\vc x)
\]
where $\psi_l$ are the base functions of the admissible space  $V_{m}$.

The optimality conditions are:
\[
   0  = \frac12\prtl_{\vc m} \Phi = \tn J[\vc m]^T \tn D^2 (\vc F[\vc m] - \overline{\vc F}) + \beta \tn A (\vc m - \vc m_0)
\]
where $\tn J=\prtl_{\vc m} \vc F$ is the sensitivity matrix $I\times L$ and $\tn A$ is a regularization matrix $L\times L$:
\[
   A_{l, k} = \int_\Omega \psi_l \psi_k + \grad \psi_l^T \tn K \grad \psi_k \d \vc x.
\]


Then we expand $\vc F$ as:
\[
    F[\vc m  + \vc d] = F[\vc m] + \tn J[\vc m] \vc d + \mathcal O(\vc d^2).
\]

Once $m_k$ is computed we determine the search direction $\vc d_{k}$ by solving the linear system:
\begin{equation}
   \label{eq::step-system}
   \Big(\tn J_k^T \tn D \tn J_k + \beta_k \tn A \Big) \vc d =
   -\tn J_k^T(\vc F_k - \overline{\vc F}) - \beta_k \tn A (\vc m_k - \vc m_0).
\end{equation}

Finally we obtain following general algorithm:

{\bf Algorithm 1}
\begin{enumerate}
 \item Set $m_0$ for $k=0$.
 \item Evaluate $\vc F_k$ and $\tn J_k$ for $\vc m_k$ (see \ref{sec:evaluation}).
 \item Determine $\beta_k$ (see \ref{sec:beta_choice})
 \item Setup and solve \eqref{eq::step-system} for $\vc d_k$
 \item Find $\lambda_k$ that satisfy suficient decrerase condition (see \ref{sec:linesearch}), set:
    $\vc m_{k+1} = \vc m_k + \lambda_k \vc d_k$
 \item If no convergence criteria is satisfied (see \ref{sec:conv_crit}) continue to 2.
\end{enumerate}

\subsection{Forward model evaluation}
\label{sec:evaluation}
Every iteration of Algorithm 1 request evaluation of the sensitivity matrix $\vc J[m_k]$ and one or more evaluations of
the forward model $\vc F[m_{k,\lambda}]$. In the case of electrical resistivity (ER) every element of $\vc F[\vc m]$
corresponds to a single measurement $i \in I$, given by the current $I$ and positions of points $C1$, $C2$, $P1$, $P2$.
In particular:
\[
    F_i[m] = u^i(\vc x^i_{P1}) - u^i(\vc x^i_{P2}),\ i\in I,
\]
where $u^i[m]$ is the solution of:
\[
  -\div \big(m \grad u^i \big) = I (\delta_{C1} - \delta_{C2}).
\]

The solution $u$ is replaced by the numerical solution $\tilde u$,
\[
    \tilde u(\vc x) = \sum_{j=1}^n U_j \phi(\vc x).
\]
Where the vector $\vc U$ is a solution of a system:
\[
    \tn L[\vc \sigma] \vc U_i = \vc b_i
\]
with
\[
    \tn L = \sum_{l=1}^L \vc m \tn L^l
\]
and
\[
    b_{i,j} = I(\phi_j(\vc x^i_{C1}) - \phi_j(\vc x^i_{C2})).
\]
Introducing similarly
\[
 c_{i,j} = \phi_j(\vc x^i_{P1}) - \phi_j(\vc x^i_{P2}),
\]
we can use matrices $\tn B$ and $\tn C$ to obtain:
\[
    \vc F[\vc m] = \tn C^T \tn L[\vc m]^{-1} \tn B.
\]
However this assume same discretization (mesh) for every measurement. Computing a single inverse
may seem attractive, but for large systems (fine mesh necessary if it has to common to all meassurements)
it is better to use a multigrid to compute $\vc F$ in time $\mathcal O(L\times n)$ where $n$ is average number of unknowns
in the refined grid for a single measurement:
\[
    F_i[\vc m] = \vc c_i^T \vc u_i, \quad \tn L_i[\vc m]\vc u_i =  \vc b_i.
\]

Differenciating this by $\vc m$ we get for $\tn J$:
\[
    J_{i,l} = \vc c_i^T \tn L_i^{-1} \tn L_i^l \tn L_i^{-1} \vc b_i
\]
or
\[
    J_{i,l} = \vc w_i^T \tn L_i^l \vc u_i, \quad \tn L_i \vc u_i = \vc b_i, \quad \tn L_i \vc w_i = \vc c_i.
\]

For every measurement $i$ we store: the mesh, set of XFEM functions, $\vc u_{i}$, $\vc w_{i}$. Updating $m_k$ to $m_{k+1}$
leads to solving a perturbed problem:
\[
   \Big(\tn L[m_k] + \lambda \tn L[d_k]\Big) \vc u = \vc b
\]
using the matrix free approach, we can just perform a few CG iterations to update $\vc u_i$ or apply single multigird cycle.

All together the complexity of single  $\tn J$ evaluation would be about
$\mathcal O(I \times n + I \times L)$ assuming that $\tn A_i^l$ is a local matrix. The evaluation of $\vc F$ have similar complexity.
Forming the system \eqref{eq::step-system} is about $\mathcal O(I \times L \times L)$ and solution of the system is of a similar complexity.
If $I \ll L$ it would be better to solve the system in matrix free fassion since Application of the operator have complexity:
$\mathcal O(I \times L)$ with good preconditioning we can allow $I \approx 1000$, $L\approx 10000$, $n\approx 10000$, inner times outer iterations $K\approx 1000$.
Total about $10^10$ operations. Should be computable in several minutes.

\subsection{Choosing regularization parameter $\beta$}
\label{sec:beta_choice}
Commonly used methods are: GCV (generalized cross validation), $L$-curve, discrepacy principle,
and AIC/BIC (Akaike/Bayessian information criterion). First three are described and compared in \cite{farquharson_2004}
the last two are compared in \cite{burnham_2004}.

\subsubsection{Discrepancy principle}
If the misfit components are independent random variableas and true deviations $\epsilon_i$ are known the expected value for $\Phi_m \approx I$.
so we try to choose $\beta$ that make this relation hold. We start with
\[
  \beta_0 = \Phi_m[m_0]/ \Phi_r[m_0].
\]
At iteration $k$ we collect last three values of $\Phi_m$, find a fit by quadratic function $P_2(\beta)$ and solve:
\[
    P_2(\beta) = I.
\]
denoting $\beta^*$ the root closer to $\beta_{k-1}$, we set:
\[
    \beta_k = \max( c\beta_{k-1}, \beta^*).
\]
where $c$ is a decrease factor $c\in (0,1)$.

\subsubsection{GCV}
The GCV approach is described in \cite{haber_gcv_2000}. The principle is to find a parameter vector $\vc m_i[\beta]$ ommiting
the $i$-th measurement from the data set and define the cross-validation function as:
\[
  CV(\beta) = \sum_i |F_i[\vc m_i] - \overline{F}_i |^2
\]
However this function is expensive to compute. First it is proved that $CV$ can be equivalently computed without solving $I$ least-square problems.
Second, the sequence of problems in Krylov subspaces is solved and the function is minimized in the subspace of much smaller dimension. There the function minimization
is feasible. The later approach can possibly be used to incrementaly increase dimension of the $V_{\vc m}$ space.




\subsection{Linesearch}
\label{sec:linesearch}
Simplest linesearch is to find $\lambda_k$ such that:
\[
    \Phi[\sigma_{k+1}] \le \Phi[\sigma_k]
\]
However more appropriate is Armijo's rule \cite{wiki_backtracking_2017} that accepts $\lambda$ when there is sufficiant
decrease in the cost function:
\[
    \Phi[\vc \sigma_{k} + \lambda \vc d_k] \le \Phi[\sigma_k] + \mu \lambda \grad \Phi^T \vc d_k.
\]
where $\mu \in (0.01,0.3)$ is a decrease parameter.


\subsection{Convergence criteria}
\label{sec:conv_crit}




\subsection{Mesh refinement}

\subsection{Adaptivity of the $\sigma$ space}

\subsection{GREIT framework}
Electrical Impedance Tomography (EIT) applies same physics to the human body. As the physics is the same
some methods and ideas can be used in both fields. Main difference in geophysical applications is:
external domain, smaller electrodes, no electrode movement.

GREIT (Gratz consensus Reconstruction algorithm for EIT) is a linear method where a matrix $R$ is found to map measurement
$\overline{\vc F}$ vector to the field vector $\vc m$. The reconstruction matrix is found as a fit to a training set, this is possible
due to similar geometry. Similar approach could be used to tunnels, precomputing usual geometries.


\section{Questions}
Klara's prezentation:
\begin{itemize}
 \item Which BC on extenal boundary? No flow. Margine min 2x dist of electrodes
 \item How big domain is neccessary compared to the measurement area (line)?
 \item Is it usual to know devitions of the measurement? How they are determined?
\end{itemize}


\section{SimPEG notes}
\begin{itemize}
    \item $\phi_d$ - data misfit part of the cost functional
    \item $\phi_m$ - regularization part of the cost functional
    \item \verb'|proj(x-g)-x|' - like: $|\vc x_1 - \vc x_0|$ with $\vc x_1 = {\rm proj}(\vc x_0 - \grad f(\vc x_0))$.
    Projection of the search direction is used. Default is identity  projection but advanced algorithms use kind of lower
    and upper bounds for compoments.
\end{itemize}

\bibliographystyle{unsrt}
\bibliography{tomography_notes}



\end{document}
